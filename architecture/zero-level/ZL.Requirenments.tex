% TODO перенести текст в описание проекта
Пользователями экосистемы будут различные лица, как компании, так и частные пользователи. С учётом этого, экосистема может применяться в рамках следующих вариантов использования:
\begin{itemize}
   \item Для личного использования.
   \item Для коммерческого использования (представление возможностей экосистемы на коммерческой основе. Может включать в себя как предоставление готовых серверов, так и требующих самостоятельной настройки)
   \item Для применения внутри компании/организации
         \label{actor-list}
\end{itemize}
Архитектура проекта разрабатывается с учётом совмещения вышеописанных вариантов использования (преимущественно двух последних).
В рамках экосистемы пользователь может взаимодействовать с следующим типами устройств:
\begin{itemize}
   \item Слабый клиент
   \item Сильный клиент
   \item Устройство инфраструктуры (IoT устройства)
   \item Сервер (с ним взаимодействует только администратор(ы))
\end{itemize}

С учётом этого, имеется 12 различных типов пользователей (каждое из 4 устройств может использоваться в рамках одного из трёх вариантов использования), у которых могут быть различные требования к соответствующему устройству и, возможно,к экосистеме в целом.
В рамках курсового проекта будет рассмотрен лишь 3-ий вариант использования экосистемы (см \ref{actor-list}) и 2 типа устройств (IoT устройства и сильный клиент рассмотрены не будут). Раздел, посвященный требованиям к проекту, будет описан в полной мере лишь для вышеперечисленных варианта использования и типов устройств.

\tcsection{Требования к ОС}
\subsection*{Бизнес-требования}
\paragraph*{Суть проекта:}
Разработка инструмента с открытым исходным кодом, реализующего бесшовную, устройство-независимую среду пользователя, организующего устройства пользователей в единую сеть, максимально полно раскрывающую возможности взаимодействия устройств, входящих в нее.

\paragraph*{Цели проекта:}
\begin{enumerate}
   \item Создать максимально устройство-независимую среду для пользователя.
   \item Снизить затраты на обслуживание вычислительных устройств в организациях.
         \begin{enumerate}[label*={\arabic*.}]
            \item Затраты на покупку, ремонт, модернизацию вычислительных устройств.
            \item Затраты на обеспечение устройств электроэнергией (за счет регуляции его потребления).
            \item Затраты на администрирование пользовательских устройств в организации.
         \end{enumerate}
   \item Упростить создание безопасной системы из устройств
   \item Позволяет гибко настраивать поведение и перечень возможностей, доступным устройствам.
   \item Создать систему, максимально широко использующую возможности всех устройств, входящих в нее
\end{enumerate}
\subsection*{Продуктные требования}
\paragraph*{Пользовательские требования:}
\begin{enumerate}[label={\bfseries ПТ-\arabic*.}]
   \item Для пользователя клиента:
         \begin{enumerate}[label*={\bfseries\arabic*.}]
            \item Работа с приложениями
                  \begin{enumerate}[label*={\bfseries\arabic*.}]
                     \item Возможность сохранения и восстановления состояния приложения между устройствами (подробное описание в \ref{DS-appstate-description}, \ref{DS-appstate_switch}).
                     \item Возможность продолжения работы приложения после выхода из системы или длительных периодов отсутствия активности
                  \end{enumerate}
            \item Работа с данными
                  \begin{enumerate}[label*={\bfseries\arabic*.}]
                     \item Возможность создания,удаления,редактирования данных пользователя на сервере экосистемы.
                  \end{enumerate}
            \item Снижение потребления устройства пользователя за счёт переключения на более экономичные протоколы обмена данными между клиентом и сервером (осуществляется выбор наилучшей альтернативы на каждом из уровней модели \href{https://ru.wikipedia.org/wiki/%D0%A1%D0%B5%D1%82%D0%B5%D0%B2%D0%B0%D1%8F_%D0%BC%D0%BE%D0%B4%D0%B5%D0%BB%D1%8C_OSI#%D0%A4%D0%B8%D0%B7%D0%B8%D1%87%D0%B5%D1%81%D0%BA%D0%B8%D0%B9_%D1%83%D1%80%D0%BE%D0%B2%D0%B5%D0%BD%D1%8C}{OSI}. %TODO ссылка на OSI, из библиографии
            \item Информация по сбору сервером данных о деятельности пользователя и информации о его устройстве должна объявляться в явном виде.
            \item Должна быть возможность управлять объемом данных, который сервер будет о пользователе и моих устройствах собирать.
         \end{enumerate}

   \item С точки зрения администратора:
         \begin{enumerate}[label*={\bfseries\arabic*.}]
            \item Возможность сбора всех данных об всех устройствах экосистемы
            \item Возможность прозрачного контроля за сотрудниками (\ref{DS-transparent-control}).
                  \begin{enumerate}[label*={\bfseries\arabic*.}]
                     \item Возможность создания черного и белого списка приложений, доступных для запускаться
                     \item Задание ограничений сетевого трафика для разных в групповой политике
                  \end{enumerate}

            \item Возможность подключения к серверу через резервные средства удаленного доступа (РСУС). По умолчанию используется средство, описанное в \ref{DS-remote_access_default}.
            \item Возможность чтения/записи конфигурации всех компонентов экосистемы и об экосистеме в целом
            \item Возможность завершать зависшие приложения вручную/автоматически
            \item Возможность получения актуальной информации о нагрузке на сервер
            \item Возможность добавлять администраторов разного уровня доступа к чтению/записи данных и конфигурации экосистемы
         \end{enumerate}
\end{enumerate}

\paragraph*{Атрибуты качества:}
\begin{enumerate}[label={\bfseries АК-\arabic*.}]
   \item Экосистема не должна запускаться дольше 20 минут при любой конфигурации
   \item Возможность обмениваться файлами с другими пользователями экосистемы без значимого падения производительности (чтение/запись файлов происходит также быстро, как и у своих файлов)
\end{enumerate}

\paragraph*{Ограничения:}
\begin{enumerate}[label={\bfseries О-\arabic*.}]
   \item Требуется наличие возможности конфигурации экосистемы.
   \item Конфигурацию экосистемы может задавать как пользователь (пользовательскую конфигурацию), так и администратор экосистемы (групповую политику).
   \item Допустимо наличие параметров конфигурации, которые доступны для редактирования и пользователем, и администратором.
         \label{R-shared_parameters}
   \item Параметры, заданные администратором и соответствующие \ref{R-shared_parameters}, имеет б\'{о}льший приоритет, чем те, что заданы пользователем.
         \label{R-parameters_types}
   \item В случае появления конфликта между параметрами, описанными в \ref{R-shared_parameters}, он разрешается всегда в пользу администратора.
   \item Пользователь может задавать конфигурацию только тех параметров, которые разрешены администратором либо выбор их значений делегирован пользователю.
   \item Конфигурация пользователя соответствует следующим требованиям:
         \begin{enumerate}[label*={\bfseries\arabic*.}]
            \item Существует 2 уровней конфигураций, каждый из которых допускается изменять:
                  \begin{itemize}
                     \item Конфигурация, общая для множества устройств (общая)
                     \item Конфигурация, уникальная для отдельно взятого устройства (локальная)
                           \label{UR-config_levels}
                  \end{itemize}
            \item Допускается существование конфигурации, принадлежащей к обоим уровням одновременно (см. \ref{UR-config_levels})
                  \label{UR-config_multilevel}
            \item Допускаются различия между локальной и общей конфигурациями (см. \ref{UR-config_multilevel})
            \item Локальная конфигурация имеет б\'{о}льший приоритет над общей.
            \item По умолчанию уровня конфигураций 2 - для всех устройств пользователя и отдельного из них
            \item Пользователь может создавать свои уровни конфигураций в расширенном режиме конфигурации
            \item Распределение устройств по уровням может происходить в различных режимах.
                  \begin{itemize}
                     \item В ручном режиме
                     \item В автоматическом режиме (основанном на правилах, которые задаёт пользователь)
                  \end{itemize}
         \end{enumerate}
   \item Групповая политика (задается лицом, описанным в \ref{R-parameters_types}) соответствует следующим требованиям:
         \begin{enumerate}[label*={\bfseries\arabic*.}]
            \item Задание групповых политик и конфигураций для групп пользователей
            \item Включение произвольного количества сотрудников в групповую политику
                  \begin{itemize}
                     \item В ручном режиме (администратор самостоятельно задает список сотрудников)
                     \item В автоматическом режиме (администратор задает правила, по которым политика к сотруднику применяется автоматически)
                  \end{itemize}
            \item Применение нескольких групповых политик, задающих разные конфигурации одного и того же параметра к одному и тому же сотруднику - недопустимо.
         \end{enumerate}
         % TODO перечисление параметров, т.е что в принципе нужно настраивать.
   \item Регуляция энергопотребления устройства пользователя.
         \begin{enumerate}[label*={\bfseries\arabic*.}]
            \item Возможность задания политики энергопотребления (частный случай задания конфигурации) % TODO 1. что включает в себя эта политика?
         \end{enumerate}
\end{enumerate}

\paragraph*{Детальные спецификации:}
\begin{enumerate}[label={\bfseriesДС-\arabic*.}]
   \item Сохранение и восстановление состояния подразумевает, что пользователь может продолжить работу с приложением в следующих случаях:
         \begin{itemize}
            \item При смене устройства, которым он пользуется на другое.
            \item После периода отсутствия активности (не пользуется ни одним устройством).
            \item В случае аварийного завершения работы как клиента, так и сервера (например, при отключении электроэнергии).
         \end{itemize}
         \label{DS-appstate_switch}
   \item Состояние приложения в себя включает:
         \begin{itemize}
            \item Состояние открытых файлов.
            \item Состояние сетевых соединений.
            \item Контекст процессора
            \item Состояние оперативной памяти
         \end{itemize}
         \label{DS-appstate-description}
   \item В качестве РСУС по умолчанию используется SSH. \label{DS-remote_access_default}
   \item Прозрачный контроль - это такой вид контроля над процессом взаимодействия объектов, когда одна или более сторон взаимодействия не знают о наличии контроля над процессом их взаимодействия до тех пор, пока не нарушат ограничения, примененные к их процессу взаимодействия. Примеры: \label{DS-transparent-control}
   \begin{itemize}
      \item Пользователю запрещено посещать сайт \url{vk.com}, при этом разрешено посещать другие. Он не будет знать о фильтрации трафика (либо ему доступны только косвенные признаки в виде небольшой задержки получения/отправки пакетов), пока не попытается зайти на данный сайт, после чего ему будет отказано.
      \item Политикой безопасности всем компонентам запрещено менять конфигурацию друг друга, и лишь ЦУЭ доступна данная возможность. Компоненты беспрепятственно коммуницируют друг с другом, пока не нарушат данное ограничение, после чего компонент безопасности откажет запрашивающему в редактировании конфигурации.
   \end{itemize}
\end{enumerate}


% \(\bullet\) Переключение камеры и прочего АО при переключении активного устройства (при переключении должна подхватиться камера с активного устройства)

% Единый интерфейс и возможности настройки всех принадлежащих пользователю устройств (наличие общих настроек)

% - Для корпоративных клиентов:
% ~~~~~~- Для коммерческих/частных клиентов:
% \begin{enumerate}[label*={\bfseries\arabic*}]
%    \item возможность выгрузить нужные приложения на сервер и работать с ними оттуда
%    \item возможность хранить данные на сервере так, чтобы они были недоступны для чтения администратором
% \end{enumerate}
