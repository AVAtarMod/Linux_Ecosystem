\tcsection{Требования к ОС}
\subsection*{Бизнес-требования}
\paragraph*{Суть проекта:}
Разработка инструмента с открытым исходным кодом, реализующего бесшовную, устройство-независимую среду пользователя, организующего устройства пользователей в единую сеть, максимально полно раскрывающую возможности взаимодействия устройств, входящих в нее.

\paragraph*{Цели проекта:}
\begin{enumerate}
   \item Создать максимально устройство-независимую среду для пользователя.
   \item Снизить затраты на обслуживание вычислительных устройств в организациях.
         \begin{enumerate}[nosep,label*={\arabic*.}]
            \item Затраты на покупку, ремонт, модернизацию вычислительных устройств.
            \item Затраты на обеспечение устройств электроэнергией (за счет регуляции его потребления).
            \item Затраты на администрирование пользовательских устройств в организации.
         \end{enumerate}
   \item Упростить создание безопасной системы из устройств
   \item Позволяет гибко настраивать поведение и перечень возможностей, доступным устройствам.
   \item Создать систему, максимально широко использующую возможности всех устройств, входящих в нее
\end{enumerate}
\subsection*{Продуктные требования}
\paragraph*{Пользовательские требования:}
\begin{enumerate}[label={\bfseries ПТ-\arabic*.}]
   \item Для пользователя слабого клиента:
         \begin{enumerate}[label*={\bfseries\arabic*.}]
            \item Работа с приложениями.
                  \begin{enumerate}[nosep,label*={\bfseries\arabic*.}]
                     \item Сохранение и восстановление состояния приложения между устройствами (подробное описание в \ref{DS-appstate-description}, \ref{DS-appstate_switch}).
                     \item Продолжение работы приложения после выхода из системы или длительных периодов отсутствия активности.
                  \end{enumerate}
            \item Работа с данными.
                  \begin{enumerate}[nosep,label*={\bfseries\arabic*.}]
                     \item Создание, удаление, редактирование данных пользователя на сервере экосистемы.
                  \end{enumerate}
            \item Перечень собираемой (сервером) информации о деятельности пользователя и о его устройстве должен объявляться в явном виде.
                  % \item Редактирование перечня собираемой (сервером) информации о деятельности пользователя и о его устройстве.
                  % TODO добавить в конфигурацию
         \end{enumerate}
         
   \item С точки зрения администратора:
         \begin{enumerate}[nosep,label*={\bfseries\arabic*.}]
            \item Сбор данных об всех устройствах экосистемы.
            \item Прозрачный контроль за пользователями (\ref{DS-transparent-control}).
                  \begin{enumerate}[nosep,label*={\bfseries\arabic*.}]
                     \item Создание черного и белого списка приложений, доступных для запускаться.
                     \item Задание ограничений сетевого трафика для пользователей.
                  \end{enumerate}
            \item Подключение к серверу через резервные средства удаленного доступа (РСУС). По умолчанию используется средство, описанное в \ref{DS-remote_access_default}.
            \item Чтение/запись конфигурации всех компонентов экосистемы и об экосистеме в целом.
            \item Получение актуальной информации о нагрузке на сервер.
            \item Добавление администраторов разного уровня доступа относительно чтения/записи данных и конфигурации экосистемы.
         \end{enumerate}
\end{enumerate}

\paragraph*{Атрибуты качества:}
\begin{enumerate}[label={\bfseries АК-\arabic*.}]
   \item Сервер экосистемы должен переходить из состояния "отключен" в состояние "готов устанавливать соединения с клиентами" не \(\leq\) за 20 минут на конфигурации, аналогичной следующей:
         \begin{itemize}
            \item ЦП: Intel Core i3-10110
            \item ПЗУ: Скорость на чтение: 10 МБ/сек последовательно, 2МБ/сек случайно; скорость на запись: 10 МБ/сек последовательно, 2МБ/сек случайно.
            \item ОЗУ: 5ГБ (доступной) серверному ПО экосистемы.
         \end{itemize}
   \item Падение производительности при обмене файлами с другими пользователями экосистемы \(\leq\) 10\%.
\end{enumerate}

\paragraph*{Ограничения:}
\begin{enumerate}[label={\bfseries О-\arabic*.}]
   \item Требуется наличие возможности конфигурации экосистемы.
   \item Конфигурацию экосистемы может задавать как пользователь (пользовательскую конфигурацию), так и администратор экосистемы (групповую политику).
   \item Допустимо наличие параметров конфигурации, которые доступны для редактирования и пользователем, и администратором.
         \label{R-shared_parameters}
   \item Параметры, заданные администратором и соответствующие \ref{R-shared_parameters}, имеет б\'{о}льший приоритет, чем те, что заданы пользователем.
         \label{R-parameters_types}
   \item В случае появления конфликта между параметрами, описанными в \ref{R-shared_parameters}, он разрешается всегда в пользу администратора.
   \item Пользователь может задавать конфигурацию только тех параметров, которые разрешены администратором либо выбор их значений делегирован пользователю.
   \item Конфигурация пользователя соответствует следующим требованиям:
         \begin{enumerate}[nosep,label*={\bfseries\arabic*.}]
            \item Существует 1 уровень конфигурации, доступный для просмотра и редактирования:
                  \begin{itemize}
                     \item Конфигурация, применяемая к отдельно взятому устройству (локальная).
                           \label{UR-config_levels}
                  \end{itemize}
         \end{enumerate}
   \item Групповая политика (задается лицом, описанным в \ref{R-parameters_types}) соответствует следующим требованиям:
         \begin{enumerate}[nosep,label*={\bfseries\arabic*.}]
            \item Задание групповых политик и конфигураций для групп пользователей.
            \item Включение произвольного количества пользователей в групповую политику.
                  \begin{itemize}[nosep]
                     \item В ручном режиме (администратор самостоятельно задает список пользователей).
                     \item В автоматическом режиме (администратор задает правила, по которым политика к пользователю применяется автоматически).
                  \end{itemize}
            \item Применение нескольких групповых политик, задающих разные конфигурации одного и того же параметра к одному и тому же пользователя - недопустимо.
         \end{enumerate}
         % \item Регуляция энергопотребления устройства пользователя.
         %       \begin{enumerate}[nosep,label*={\bfseries\arabic*.}]
         %          \item Возможность задания политики энергопотребления (частный случай задания конфигурации).
         %       \end{enumerate}
         % TODO перенести в политику компонента
   \item Перечень параметров, доступных для редактирования, полностью зависит от набора компонентов на клиенте и сервере.
\end{enumerate}

\paragraph*{Детальные спецификации:}
\begin{enumerate}[label={\bfseriesДС-\arabic*.}]
   \item Сохранение и восстановление состояния подразумевает, что пользователь может продолжить работу с приложением в следующих случаях:
         \begin{itemize}
            \item При смене устройства, которым он пользуется на другое.
            \item После периода отсутствия активности (не пользуется ни одним устройством).
            \item В случае аварийного завершения работы как клиента, так и сервера (например, при отключении электроэнергии).
         \end{itemize}
         \label{DS-appstate_switch}
   \item Состояние приложения в себя включает:
         \begin{itemize}
            \item Состояние открытых файлов,
            \item Состояние сетевых соединений,
            \item Контекст процессора,
            \item Состояние оперативной памяти.
         \end{itemize}
         При этом состояние приложения хранится на сервере.
         \label{DS-appstate-description}
   \item В качестве РСУС по умолчанию используется SSH. \label{DS-remote_access_default}
   \item Прозрачный контроль - это такой вид контроля над процессом взаимодействия объектов, когда одна или более сторон взаимодействия не знают о наличии контроля над процессом их взаимодействия до тех пор, пока не нарушат ограничения, примененные к их процессу взаимодействия. Примеры: \label{DS-transparent-control}
         \begin{itemize}
            \item Пользователю запрещено посещать сайт \url{vk.com}, при этом разрешено посещать другие. Он не будет знать о фильтрации трафика (либо ему доступны только косвенные признаки в виде небольшой задержки получения/отправки пакетов), пока не попытается зайти на данный сайт, после чего ему будет отказано.
            \item Политикой безопасности всем компонентам запрещено менять конфигурацию друг друга, и лишь ЦУЭ доступна данная возможность. Компоненты беспрепятственно коммуницируют друг с другом, пока не нарушат данное ограничение, после чего компонент безопасности откажет запрашивающему в редактировании конфигурации.
         \end{itemize}
\end{enumerate}
