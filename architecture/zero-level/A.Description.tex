Экосистема устройств имеет следующую архитектуру: сетевая операционная система (ОС) в совокупности с системными утилитами и вспомогательным ПО (middleware).
Архитектура экосистемы состоит из объектов 3 уровней -- \(0,1,2\) где 2 уровень -- наименьший. Данное деление обусловлено сложностью программного продукта и необходимостью рассмотрения и поведения системы с разных сторон. Структурно, объект уровня \(N-1\) -- неотъемлемая часть объекта уровня \(N\). В таблице \ref{table-objects} перечислены все их уровни.
\begin{table}[ht]
   \centering
   \begin{tabularx}{\textwidth}{|>{\centering}p{.09\textwidth}|p{.15\textwidth}|X|}
      \cline{1-3}
      Номер уровня & Название  & Описание \\ \hline
      0 & Сервер & Самое мощное вычислительное устройство экосистемы. Является ее связующим звеном (без него экосистема "распадется" на отдельные устройства). Умеет хранить данные пользователя, обрабатывать запросы с устройств. Позволяет управлять всей экосистемой. Подразумевается, что он постоянно запущен и подключен к сети. \\ \hline
      0 & Слабый клиент & Слабое вычислительное устройство (планшет, слабый ПК и т.д), которое пользуется возможностями сервера или толстого клиента (если толстый клиент ближе и может взаимодействовать с ним). Обязательно обладает устройствами ввода-вывода, сетевой картой, собственным ПЗУ минимального объема, ОЗУ, ЦП, графическим процессором. При необходимости может эмулироваться на сервере. \\ \hline
      0 & Пользователь & Человек, использующий возможности экосистемы через клиента или устройство инфраструктуры. С точки зрения управления экосистемой может быть обычным пользователем или администратором. \\ \hline
      1 & Компонент экосистемы & ПО экосистемы, часть ОС, реализующая одну из ее функций. Обязательно присутствие как на клиенте, так и на сервере. \\ \hline
      1 & Компонент экосистемы & ПО экосистемы, часть ОС, реализующая одну из ее функций. Обязательно присутствие как на клиенте, так и на сервере. \\ \hline
      2 & Часть компонента экосистемы & ПО экосистемы, реализующее определенную функцию компонента. \\ \hline
   \end{tabularx}
   \caption{Перечень объектов экосистемы}
   \label{table-objects}
\end{table}

Таким образом, ОС имеет модульную и клиент-серверную архитектуру. Это означает, что, как на сервере, так и на клиенте работу ОС поддерживают компоненты, общающиеся как между друг-друг другом, так и с соответствующей ему серверной/клиентской стороной.
Каждый из компонентов отвечает за отдельную функцию ОС.

\paragraph{Обоснование:} 
Реализация каждой из функций ОС из-за своей сложности и объемности подразумевает декомпозицию всей ОС на части, выполняющие отдельные функции; ОС разрабатывается с учетом высокой загруженности серверной части, поддержки масштабирования экосистемы; существует потребность в параллельной/одновременной работе разных модификациях одних и тех же компонентов для обслуживания запросов различных классов клиентов.
Все выше перечисленное приводит к необходимости разбиения ОС на компоненты на программном уровне.
