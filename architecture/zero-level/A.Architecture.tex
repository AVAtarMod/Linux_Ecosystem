Выбор архитектуры проекта:
Экосистема устройств имеет следующую архитектуру: сетевая операционная система (ОС) в совокупности с системными утилитами и вспомогательным ПО (middleware)

## Об архитектуре ОС
ОС имеет модульную и клиент-серверную архитектуру. Это означает, что, как на сервере, так и на клиенте работу ОС поддерживают компоненты, общающиеся как между друг-друг другом, так и с соответствующей ему серверной/клиентской стороной.
Каждый из компонентов отвечает за отдельную функцию ОС (например, за хранение данных, ограничением доступа, трансляцию данных и т.д)

\paragraph{Обоснование.} 
Реализация каждой из функций ОС из-за своей сложности и объемности подразумевает декомпозицию всей ОС на части, выполняющие отдельные функции; ОС разрабатывается с учетом высокой загруженности серверной части, поддержки масштабирования экосистемы; существует потребность в параллельной/одновременной работе разных модификациях одних и тех же компонентов для обслуживания запросов различных классов клиентов.
Все выше перечисленное приводит к необходимости разбиения ОС на компоненты на программном уровне.

## Архитектура компонентов ОС
Компоненты имеют преимущественно монолитную архитектуру, могут поставляться с модулями для ядра ОС для повышения производительности критических функций (например, фильтрация запросов между компонентами компонентом "Безопасность" осуществляется с участием модуля ядра ОС). 

Компонент "Статистика" имеет модульную архитектуру. Модули называются плагинами. 

\paragraph{Обоснование.} 
Способов обработки и анализа статистических данных - неограниченное количество. При этом, потребность пользователя в них может меняться со временем, они могут активно развиваться и изменяться.
С другой стороны, процесс сбора статистических данных (процесс создания статистической выборки) по своей сути достаточно однообразен, что позволяет сделать эту функцию одинаковой для всех плагинов. Так как процесс сбора статистических данных однообразен, естественным выбором является ее организация как функции, независимой от плагинов.

Все вышеперечисленное приводит к решению реализовывать обработку и анализ какой бы то ни было информации в плагинах, тогда как сам компонент может отвечать за сбор необработанных статистических данных.  

## Об архитектуре программных инструментов(ПИ)
В основном, ПИ будут относительно простыми и поэтому будут иметь монолитную архитектуру.
