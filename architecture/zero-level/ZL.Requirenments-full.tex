% TODO перенести текст в описание проекта
\Large{АКТУЛЬНАЯ ВЕРСИЯ ТРЕБОВАНИЙ ПРОТОТИПА НАХОДИТСЯ В ФАЙЛЕ \(ZL.Requirenments.tex\)}

Пользователями экосистемы будут различные лица, как компании, так и частные пользователи. С учётом этого, экосистема может применяться в рамках следующих вариантов использования:
\begin{itemize}
   \item Для личного использования.
   \item Для коммерческого использования (представление возможностей экосистемы на коммерческой основе. Может включать в себя как предоставление готовых серверов, так и требующих самостоятельной настройки)
   \item Для применения внутри компании/организации
         \label{actor-list}
\end{itemize}
Архитектура проекта разрабатывается с учётом совмещения вышеописанных вариантов использования (преимущественно двух последних).
В рамках экосистемы пользователь может взаимодействовать с следующим типами устройств:
\begin{itemize}
   \item Слабый клиент
   \item Сильный клиент
   \item Устройство инфраструктуры (IoT устройства)
   \item Сервер (с ним взаимодействует только администратор(ы))
\end{itemize}
С учётом этого, имеется 12 различных типов пользователей (каждое из 4 устройств может использоваться в рамках одного из трёх вариантов использования), у которых могут быть различные требования к соответствующему устройству и, возможно,к экосистеме в целом.
В рамках курсового проекта будет рассмотрен лишь 3-ий вариант использования экосистемы (см \ref{actor-list}) и 2 типа устройств (IoT устройства и сильный клиент рассмотрены не будут). Раздел, посвященный требованиям к проекту, будет описан в полной мере лишь для вышеперечисленных варианта использования и типов устройств.
\tcsection{Требования к ОС}
Пользовательские требования:
% TODO автоматическая генерация ссылок
% TODO проставить перекрестные ссылки
\begin{enumerate}[label={\bfseries ПТ-\arabic*}]
   \item Для пользователя тонкого клиента:
         \begin{enumerate}[label*={\bfseries.\arabic*}]
            \item Работа с приложениями
                  \begin{enumerate}[label*={\bfseries.\arabic*}]
                     \item Возможность сохранения состояния приложения между устройствами (подробное описание в \ref{DS1.1}) %TODO DS о описании состояния, как оно переносится
                     \item Возможность переноса приложения на сервер (также см. \ref{R1.1}) % TODO R об случае, когда архитектуры не совпадают, а также когда приложение под архитектуру сервера не существует
                           % \label{UR-1.2.0}
                     \item Динамическое переключение между устройствами ввода/вывода на основе текущего активного устройства и его возможностей(также см. \ref{R1.2}) %TODO R об ограничениях в случае когда у активного клиента отсутствуют аналогичные устройства
                     \item Возможность продолжения работы приложения после выхода из системы или длительных периодов отсутствия активности
                  \end{enumerate}
            \item Конфигурация экосистемы
                  \begin{enumerate}[label*={\bfseries.\arabic*}]
                     \item Наличие 2 уровней конфигураций, каждый из которых допускается изменять:
                           \begin{itemize}
                              \item Конфигурация, общая для множества устройств (общая)
                              \item Конфигурация, уникальная для отдельно взятого устройства (локальная)
                                    \label{UR1.2.1}
                           \end{itemize}
                     \item Допускается существование конфигурации, принадлежащей к обоим уровням одновременно (см. \ref{UR1.2.1})
                           \label{UR1.2.2}
                     \item Допускаются различия между локальной и общей конфигурациями (см. \ref{UR1.2.2})
                     \item Локальная конфигурация имеет б\'{о}льший приоритет над общей.
                     \item По умолчанию уровня конфигураций 2 - для всех устройств пользователя и отдельного из них
                     \item Пользователь может создавать свои уровни конфигураций в расширенном режиме конфигурации
                     \item Распределение устройств по уровням может происходить в различных режимах.
                           \begin{itemize}
                              \item В ручном режиме
                              \item В автоматическом режиме (основанном на правилах, которые задаёт пользователь)
                           \end{itemize}
                  \end{enumerate}
            \item Работа с данными
                  \begin{enumerate}[label*={\bfseries.\arabic*}]
                     \item Возможность создания,удаления,редактирования данных пользователя на сервере экосистемы.
                     \item Возможность передачи данных с устройств (USB флеш-накопитель, внешний SSD), подключенных к клиенту на сервер и обратно.
                  \end{enumerate}
            \item Регуляция энергопотребления устройства пользователя
                  \begin{enumerate}[label*={\bfseries.\arabic*}]
                     \item Возможность задания политики энергопотребления (частный случай задания конфигурации) % TODO: что включает в себя эта политика?
                     \item Снижение вычислительной нагрузки на устройство пользователя за счёт переноса приложений на сервер (см \ref, {UR1.2.0} %TODO поправить ссылки
                           )
                     \item Снижение потребления за счёт переключения на более экономичные протоколы обмена данными между клиентом и сервером (осуществляется выбор наилучшей альтернативы на каждом из уровней модели OSI %TODO ссылка на OSI, из библиографии)
                     \item Допускается использование сильного клиента в роли сервера (с учётом \ref{R1.3} %TODO R о максимальном количестве тонких клиентов у сильного
                           ) в целях сокращения энергопотребления слабого клиента за счёт повышения нагрузки на сильный клиент (см. \ref{DS1.2} %TODO DS о сокращении мощности модема клиента)
                     \item
                  \end{enumerate}
            \item Возможность обмениваться файлами с другими пользователями экосистемы без значимого падения производительности (чтение/запись файлов происходит также быстро, как и у своих файлов)
            \item Информация по сбору сервером данных о моей деятельности и информации о моем устройстве, должна объявляться в явном виде
            \item Должна быть возможность управлять объемом данных, который сервер будет о пользователе и моих устройствах собирать
         \end{enumerate}

   \item С точки зрения администратора:
         \begin{enumerate}[label*={\bfseries.\arabic*}]
            \item Возможность сбора всех данных об всех устройствах экосистемы
            \item Задание групповых политик и конфигураций для групп пользователей
            \item Включение произвольного количества сотрудников в групповую политику
                  \begin{itemize}
                     \item В ручном режиме (администратор самостоятельно задает список сотрудников)
                     \item В автоматическом режиме (администратор задает правила, по которым политика к сотруднику применяется автоматически)
                  \end{itemize}
            \item Применение нескольких групповых политик, задающиих разные конфигурации одного и того же параметра к одному и тому же сотруднику - недопустимо
            \item Возможность прозрачного контроля за сотрудниками (\ref{R1.2.2}) %TODO R о понятии "прозрачного контроля"
                  \begin{enumerate}[label*={\bfseries.\arabic*}]
                     \item Возможность создания черного и белого списка приложений, доступных для запускаться
                     \item Задание ограничений сетевого трафика для разных в групповой политике
                  \end{enumerate}
            \item Отсутствие необходимости обслуживать тонкие клиенты: купил-> настроил через сервер->забыл
            \item Сервер должен уметь автоматически/вручную обновлять прошивки тонких и толстых клиентов
            \item Экосистема не должна запускаться дольше 20 минут при любой конфигурации
            \item Возможность предсказывания проблем с железом устройств (составление прогноза времени выхода из строя АО клиентов и серверов)
            \item Возможность оповещения о проблемах в экосистеме на почту, PUSH в приложение и т.д
            \item Возможность подключения к серверу через РСУС (ssh по умолчанию)
                  % TODO: расписать относ. железа, привести конкретные конф-ции
            \item Экосистема должна устанавливаться за 30 минут (без учета времени, затраченного на
                  ожидание ввода пользователя) на следующей конфигурации оборудования:
                  \begin{itemize}
                     \item ЦП: Intel Core i3-10110
                     \item ПЗУ: Скорость на чтение: 30 МБ/сек последовательно, 7МБ/сек случайно
                     \item ПЗУ: Скорость на запись: 20 МБ/сек последовательно, 5МБ/сек случайно
                     \item ОЗУ: 5ГБ (доступной)
                  \end{itemize}
            \item Возможность чтения/записи конфигурации всех компонентов экосистемы и об экосистеме в целом
            \item Возможность фильтрации трафика пользователей (блокировать доступ к VK и прочим сайтам, или лимитировать его)
            \item Возможность завершать зависшие приложения вручную/автоматически
            \item Возможность получения актуальной информации о нагрузке на сервер
            \item Возможность добавлять администраторов разного уровня доступа к чтению/записи данных и конфигурации экосистемы
         \end{enumerate}
\end{enumerate}

\paragraph*{Детальные спецификации:}
\begin{enumerate}[label={\bfseries ДС-\arabic*}]
   \item Сохранение и восстановление состояния подразумевает, что пользователь может продолжить работу с приложением в следующих случаях:
         \begin{itemize}
            \item При смене устройства, которым он пользуется на другое.
            \item После периода отсутствия активности (не пользуется ни одним устройством).
            \item В случае аварийного завершения работы как клиента, так и сервера (например, при отключении электроэнергии).
         \end{itemize}
         \label{DS-appstate_switch}
   \item Состояние приложения в себя включает:
   \begin{itemize}
      \item Состояние открытых файлов.
      \item Состояние сетевых соединений.
      \item Контекст процессора
      \item Состояние оперативной памяти
      \item Другие составные (если имеются), необходимые для восстановления работы приложения
   \end{itemize}
   \item В качестве РСУС по умолчанию используется SSH. \label{DS-remote_access_default}
\end{enumerate}

\(\bullet\) Переключение камеры и прочего АО при переключении активного устройства (при переключении должна подхватиться камера с активного устройства)

Единый интерфейс и возможности настройки всех принадлежащих пользователю устройств (наличие общих настроек)

- Для корпоративных клиентов:
~~~~~~- Для коммерческих/частных клиентов:
\begin{enumerate}[label*={\bfseries\arabic*}]
   \item возможность выгрузить нужные приложения на сервер и работать с ними оттуда
   \item возможность хранить данные на сервере так, чтобы они были недоступны для чтения администратором
\end{enumerate}
