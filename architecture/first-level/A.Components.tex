Каждый из объектов 0 уровня экосистемы состоит из объектов 1 уровня.
% ! Определение в файле A.Description.tex
\tcsection{Компоненты экосистемы}
В соответствии с определением, компонент реализует функцию экосистемы, требующую наличие логики как на клиентах, так и на сервере. В таблице \ref{table-components} перечислены те компоненты, которые составляют основу экосистемы, вместе с их описанием. В данной работе будут реализованы 
\begin{table}[ht]
   \centering
   \begin{tabularx}{\textwidth}{|>{\centering}p{.2\textwidth}|X|}
      \cline{1-2}
      Название & Назначение \\
      \hline
      Статистика & Ведение и обработка статистики за устройствами и пользователями. Анализ полученных данных для различных целей. Предоставление интерфейса для анализа, экспорта и импорта статистики сторонними модулями или программными инструментами экосистемы. \\
      \hline
      Хранение данных & Хранение данных экосистемы, управление и гибкое применение технологий повышения надежности хранения данных (RAID и пр.). Ведение резервных копий данных экосистемы. Оптимизация скорости доступа к данным. \\
      \hline
      Поддержка аппаратного обеспечения (АО) устройств & Предоставление остальным компонентам экосистемы интерфейс взаимодействия с АО устройств. Управление и конфигурация драйверов устройств. Обеспечение целостности АО экосистемы (прозрачное* переключение на резервное АО в т.ч.). Управление сетевыми подключениями уровня клиент-сервер \\
      \hline
      Управление пользовательской средой & Организация бесшовной среды для пользователя:
      Прозрачное переключение устройств ввода-вывода на уровне приложений
      Трансляция приложений и пользовательского интерфейса ОС клиентам.
      'Умное' использование доступного для устройств АО (поддержка совместного применения всего АО, доступного устройствам экосистемы) \\
      \hline
      Регистрация объектов & Организация процесса регистрации и устройств, пользователей в экосистеме. Организация процесса модификации и удаления данных об устройствах и пользователях экосистемы. Организация авторизации устройств и пользователей. \\
      \hline
   \end{tabularx}
   \caption{Перечень базовых компонентов экосистемы}
   \label{table-components}
\end{table}
