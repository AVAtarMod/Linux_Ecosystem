% ! Определение в файле A.Description.tex
\paragraph*{Объекты 1 уровня прототипа экосистемы.}
В соответствии с таблицей \ref{table-objects}, на 1 уровне находятся компоненты и программные инструменты экосистемы. В таблице \ref{table-components} перечислены компоненты, реализующие базовые функции экосистемы, вместе с их описанием.
\begin{table}[ht]
   \centering
   \begin{tabularx}{\textwidth}{|>{\centering}p{.2\textwidth}|X|}
      \cline{1-2}
      Название & Назначение \\
      \hline
      Статистика & Ведение и обработка статистики за устройствами и пользователями. Анализ полученных данных для различных целей. Предоставление интерфейса для анализа, экспорта и импорта статистики сторонними модулями или программными инструментами экосистемы. \\
      \hline
      Хранение данных & Установка компонентов экосистемы, конфигурация хранения данных, взаимодействие с СУБД. \\
      \hline
      Поддержка аппаратного обеспечения (АО) устройств & Предоставление остальным компонентам экосистемы интерфейс взаимодействия с АО устройств. Управление и конфигурация драйверов устройств. Обеспечение целостности АО экосистемы (прозрачное* переключение на резервное АО в т.ч.). Управление сетевыми подключениями уровня клиент-сервер \\
      \hline
      Управление пользовательской средой & Организация бесшовной среды для пользователя:
      Прозрачное переключение устройств ввода-вывода на уровне приложений
      Трансляция приложений и пользовательского интерфейса ОС клиентам.
      'Умное' использование доступного для устройств АО (поддержка совместного применения всего АО, доступного устройствам экосистемы) \\
      \hline
      Регистрация объектов & Организация процесса регистрации и устройств, пользователей в экосистеме. Организация процесса модификации и удаления данных об устройствах и пользователях экосистемы. Организация авторизации устройств и пользователей. \\
      \hline
   \end{tabularx}
   \caption{Перечень базовых компонентов прототипа экосистемы}
   \label{table-components}
\end{table}

Далее будет описан перечень программных инструментов экосистемы, реализующий необходимый минимум функций экосистемы.
\begin{table}[ht]
   \centering
   \begin{tabularx}{\textwidth}{|>{\centering}p{.2\textwidth}|X|}
      \cline{1-2}
      Название & Назначение \\
      \hline
      Центр управления экосистемой (ЦУЭ) & ПО, необходимое для получения сведений о экосистеме, конфигурации ее компонентов, администрирования устройств и пользователей. \\
      \hline
      Менеджер управления компонентами & ПО, выполняющее следующие задачи:
      запуск экосистемы на стороне сервера, 
      передача компонентам информации для подключения к СУБД, 
      хранение информации о запущенных компонентах (в СУБД) \\
      \hline
      СУБД & ПО, организующее хранение данных экосистемы \\
      \hline
   \end{tabularx}
   \caption{Перечень программных инструментов прототипа экосистемы}
   \label{table-systools}
\end{table}

\paragraph{Архитектура компонентов экосистемы:} 
Компоненты имеют преимущественно монолитную архитектуру, могут поставляться с модулями для ядра ОС для повышения производительности критически важных функций.
Компонент <<Статистика>> имеет модульную архитектуру. Модули называются плагинами. 

\paragraph{Обоснование:} 
Способов обработки и анализа статистических данных - неограниченное количество. При этом, потребность пользователя в них может меняться со временем, они могут активно развиваться и изменяться.
С другой стороны, процесс сбора статистических данных (процесс создания статистической выборки) по своей сути достаточно однообразен, что позволяет сделать эту функцию одинаковой для всех плагинов. Так как процесс сбора статистических данных однообразен, естественным выбором является ее организация как функции, независимой от плагинов.

Все вышеперечисленное приводит к решению реализовывать обработку и анализ какой бы то ни было информации в плагинах, тогда как сам компонент может отвечать за сбор необработанных статистических данных.  

\paragraph{Об архитектуре программных инструментов(ПИ):}
В основном, ПИ будут относительно простыми и поэтому будут иметь монолитную архитектуру.
